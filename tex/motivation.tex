Directed probabilistic graphical models can represent many real-world problems. But there are some problems, which cannot be addressed by directed but by undirected graphical models. In addition to that undirected graphs might also be easier to understand and might enable to model a problem in an simpler way.

A core advantage of undirected graphs for probabilistic modeling over for example a Bayesian network (as common directed graphical model) is the property that undirected graphs are capable of representing cycles. For example if the impact of four planets on each other in terms of gravity should be modeled there is neither a starting point nor a target point of conditional dependency among those four planets. The acceleration of each planet mainly depends on the gravity fields of the three other planets (and the distance). Indeed, the origin of undirected graphical models lie in the theoretical physics, for example in the Ising model, described in \cite{ising1925beitrag}, which discuss the impact of magnetic elementary particles on each other. Undirected graphical models are the generalization of this problem and enable to model such cyclic dependencies.

In contrast to Bayesian networks the parameterization is not often easy to understand, because, as presented later in section \ref{sec:methodology}, the parameters do no directly represent the probabilities among the nodes but rather some kind of compatibility. This property provides a huge amount of flexibility but also a more difficult handling. How those compatibilities are used is shown in section \ref{sec:application}, when discussing the applications of undirected graphical models. 

In order to confront those problems and to understand the advantages of undirected graphical models, the next section gives a basic theoretical foundation of this topic.
